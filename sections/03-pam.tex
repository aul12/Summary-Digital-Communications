\chapter{Digital \acl{pam}}
\section{Basic Principles}
\subsection{Basic Pulse Shape}
The basic pulse shape is denoted by $g(t)$. This pulse can be seen as the impulse response
of the transmit filter.

Important properties:
\begin{itemize}
    \item Spectrum of the pulse $G(f) \Laplace g(t)$
    \item Energy of the pulse:
        \begin{equation}
            E_g = \int_{-\infty}^\infty \abs{g(t)}^2 \text{d}t 
                = \int_{-\infty}^\infty \abs{G(f)}^2 \text{d}f
        \end{equation}
\end{itemize}

\subsection{Signal Constellation}
The signal constellation $\mathcal{A}$ is a set of amplitude coefficients or signal points.

Important properties:
\begin{itemize}
    \item $M = \abs{\mathcal{A}}$: cardinality (number of points)
    \item $R_M = \log_2(M)$: rate of modulation (number of bits)
\end{itemize}

\subsection{Symbol Period}
The symbol period (or duration of a modulation interval) $T$ is given via $g(t)$.

It is related to bit duration and rate of modulation by:
\begin{equation}
    T = T_b \cdot R_M
\end{equation}

\subsection{Modulation}
For every timestep (duration $T$) $R_M$ source symbols are collected from the source
signal $<q[l]>$ and mapped onto a single signal point $a[k] \in \mathcal{A}$. This
sequence of signal points $<a[k]>$ is then filtered by the LTI system defined by $g(t)$,
thus every signal point scales its version of $g(t)$ at the position of the signal point.

The transmit signal is given by:
\begin{equation}
    s(t) = \sum_{k=-\infty}^\infty a[k] g(t - k T)
\end{equation}

\subsection{Baseband and Carrier-Modulated Transmission}
In baseband no upconversion takes place, thus $s(t)$ must be real. This can be achieved
by having a real $g(t)$ and only real amplitude coefficients $a[k] \in \mathcal{A}$.

In carrier modulated transmission $s(t)$ is the \ac{ecb} signal. $g(t)$ and
$a[k] \in \mathcal{A}$ can be arbitrary, in pratice $g(t)$ is often chosen to be real.

By splitting the amplitude components in quadrature and inphase components, i.e.
$a[k] = a_I[k] + j a_Q[k]$, and assuming a real $g(t)$  we get the RF signal as:
\begin{equation}
    s_\text{RF}(t) = \sqrt{2} \left(\sum_{k=-\infty}^\infty a_I[k] g(t-kT)
        \right) \cos(2 \pi f_c t)
        - \sqrt{2} \left(\sum_{k=-\infty}^\infty a_Q[k] g(t-kT)\right)
            \sin(2 \pi f_c t)
\end{equation}

\section{Average \acl{psd} of \acl{pam} signals}
As the source sequence is drawn from a wide-sense stationary process the ampltitude
coefficients are drawn from a wide-sense stationary process as well. Thus the
\ac{acf} is given by:
\begin{equation}
    \phi_{aa}[\kappa] = E\{a[k + \kappa] a^*[k]\}
\end{equation}
and the \ac{psd} is given by ($z$-transform on the unit circle):
\begin{equation}
    \Phi_{aa}(e^{j 2 \pi f T}) = \sum_{\kappa = -\infty}^\kappa \phi_{aa}[\kappa] e^{-j 2 \pi f T \kappa}
\end{equation}

\subsection{Average \acl{acf} of $s(t)$}
Due to the clocking $s(t)$ is drawn from a cyclo stationary process.

The average \ac{acf} is given by:
\begin{equation}
    \bar{\phi}_{ss}(\tau) = \frac{1}{T} \int_0^T E\{s(t+\tau)s*(t)\} \text{d}t
\end{equation}
by substituting $s$ with the definitions we arrive at:
\begin{equation}
    \bar{\phi}_{ss}(\tau) =  \frac{1}{T} \sum_{\kappa} \phi_{aa}[\kappa] \varphi_{gg}(\tau - \kappa T)
\end{equation}
with
\begin{equation}
    \varphi_{gg}(\tau) = g(\tau) * g^*(-\tau)
\end{equation}

\subsection{Average \acl{psd} of $s(t)$}
Taking the average-\ac{psd} as the fourier transform of the average-\ac{acf} we get:
\begin{equation}
    \bar{\Phi}_{ss}(f) = \Phi_{aa}(e^{j 2 \pi f T}) \frac{\abs{G(f)}^2}{T}
\end{equation}

\subsection{Uncoded Transmission}
For uncoded transmission we assume stastically independent and equally
distributed source symbols $q[l]$. Thus the signal points keep these properties.
This simplifies the \ac{acf}:
\begin{equation}
    \phi_{aa}[\kappa] = E\{a[k+\kappa]a^*[k]\} =
    \begin{cases}
        \sigma_a^2 + \abs{m_a}^2 & \kappa = 0 \\
        \abs{m_a}^2 & \text{else}
    \end{cases}
\end{equation}
with $\sigma_a^2$ and $m_a$ as mean and variance respectively.

The \ac{psd} is then given by:
\begin{equation}
    \Phi_{aa}(e^{j 2 \pi f T}) =
        \sigma_a^2 + \frac{\abs{m_a}^2}{T} \sum_{l=-\infty}^\infty \delta(f - \frac{l}{T})
\end{equation}

The average \ac{psd} of uncoded \ac{pam} is then given by:
\begin{equation}
    \bar{\Phi}_{ss}(f) =
        \sigma_a^2 \frac{\abs{G(f)}^2}{T} + \abs{m_a}^2\frac{\abs{G(f)}^2}{T^2} \sum_{l=-\infty}^\infty \delta(f - \frac{l}{T})
\end{equation}

\subsection{Average transmit power}
For zero mean amplitude coefficients $m_a=0$ we get:
\begin{itemize}
    \item Average transmit power:
        \begin{equation}
            S_S = \sigma_a^2 \frac{E_g}{T}
        \end{equation}
    \item Average energy per symbol:
        \begin{equation}
            E_S = S_S \cdot T = \sigma_a^2 E_g
        \end{equation}
    \item Average energy per bit:
        \begin{equation}
            E_b = \frac{E_s}{R_M}
        \end{equation}
\end{itemize}

For $m_a \neq 0$ the calculations are more complicated, for a $\sqrt{\text{Nyquist}}$
pulse we get the average transmit power as:
\begin{equation}
    S_S = (\sigma_a^2 + \abs{m_a}^2) \frac{E_g}{T}
\end{equation}

\section{Choice of the Signal Constellation}
\subsection{\acl{ask}}
In baseband transmission the constellation has to be real valued, thus signal points
on the real axis are chosen.

There are two versions of \ac{ask}:
\begin{itemize}
    \item In bipolar \ac{ask} the signal points are spaced symmetrical
    
        The signal constellation is given by $\mathcal{A} = \{\pm 1, \pm 3, \ldots, \pm (M-1)\}$.

        Mean and variance are given by:
        \begin{eqnarray}
            m_a &=& 0 \\
            \sigma_a^2 &=& \frac{M^2 -1}{3}
        \end{eqnarray}
    \item In unipolar \ac{ask} the signal points are only on the positive axis

        The signal constellation is given by $\mathcal{A} = \{0, 2, \ldots, 2 \cdot (M-1)\}$

        Mean and variance are given by:
        \begin{eqnarray}
            m_a &=& M-1 \\
            \sigma_a^2 &=& \frac{M^2 -1}{3}
        \end{eqnarray}
\end{itemize}

\subsection{\acl{psk}}
\ac{psk} yields signal points of constant magnitude. The signal constellation is given
by:
\begin{equation}
    \mathcal{A} = \{e^{j 2 \pi \frac{m-1}{M}} | m \in \{1, \ldots, M\}\}
\end{equation}
Mean and variance are given by:
\begin{eqnarray}
    m_a &=& 0 \\
    \sigma_a^2 &=& 1
\end{eqnarray}

\subsection{Digital \acl{qam}}
For \ac{qam} the signal points are spaced regularly on the complex plane.
If $M=2^m$ is a square number (i.e. $m$ is even), the boundary of the constellation
is square, thus such constellations are called square constellation. Otherwise
signals points in the corner are missing, such constellations are refered to as cross
constellations.

For square constellation mean and variance are given by:
\begin{eqnarray}
    m_a &=& 0 \\
    \sigma_a^2 &=& \frac{2 (M-1)}{3}
\end{eqnarray}

\section{Choice of the Basis Pulse Shape}
\subsection{Optimum Receive Filter}
The receive signal is given by:
\begin{equation}
    r(t) = s(t) + n(t)
\end{equation}
For AWGN-Channels $n(t)$ is complex valued, white Gaussian noise with \ac{psd} $N_0$.

General setup for the receiver:
\begin{enumerate}
    \item The receive signal $r(t)$ is filtered with $h_R(t)$
    \item The filtered signal is sampled in order to obtain decision variables $d[k]$
    \item Decisions are taken based on $d[k]$, estimates of the amplitude
        coefficients $\tilde{a}[k]$ are generated (i.e. inverse mapping)
\end{enumerate}

We optimized the \ac{snr} at the receiver at detection time $T_d$, it is given by:
\begin{equation}
    \text{SNR}_d = \frac{\abs{\tilde{d}(T_d)}^2}{\sigma_{n_d}^2}
\end{equation}
the useful signal is given by
\begin{equation}
    \tilde{d}(T_d) = g(t) * h_R(t) \vert_{t=T_d} = \int_{-\infty}^\infty G(f) H_R(f)
        e^{j 2 \pi f T_d} \text{d}f
\end{equation}
and the noise power by
\begin{equation}
    \sigma_{n_d}^2 = \frac{N_0}{2} \int_{-\infty}^\infty \abs{H_R(f)}^2 \text{d}f
\end{equation}

thus the \ac{snr} is bounded by:
\begin{equation}
    \text{SNR}_d \leq \frac{E_g}{N_0/2}
\end{equation}
the equality is obtained if $\bar{H}_R(f) = {(H_R(f) e^{j 2 \pi f T_d})}^*$ is a multiple
of $G(f)$.

Thus the optimum receive filter is given by the matched filter:
\begin{equation}
    h_R(t) = \gamma g^*(T_d -t)
\end{equation}
or
\begin{equation}
    H_R(f) = \gamma G^*(f) e^{-j 2 \pi f T_d}
\end{equation}

\subsubsection{Detection Signal}
The detection signal is given by
\begin{equation}
    \tilde{d}(t) = \gamma \varphi_{gg}(t- T_d)
\end{equation}
at the sampling time we get:
\begin{equation}
    \tilde{d}(T_d) = \gamma E_b
\end{equation}

The \ac{acf} of the AWGN-\ac{ecb}-noise after the matched filter is given by:
\begin{equation}
    \phi_{n_d n_d}(\tau) = N_0 \gamma^2 \varphi_{gg}(\tau)
\end{equation}

and thus the noise power is
\begin{equation}
    \varphi_{n_d}^2 = N_0 \gamma^2 \varphi_{gg}(0)
\end{equation}

\subsubsection{Optimum receive filter}
The receive filter is parameterized by $T_d$, the receive delay. We will assume $T_d=0$,
in applications $T_d > 0$ so that the receive filter is causal.

\subsection{\acl{isi} Free Transmission - The Nyquist Criterion}
When considering multiple symbols the receive signal is given by:
\begin{equation}
    \tilde{d}[k] = \gamma \sum_\kappa a[\kappa] \varphi_{gg}((k - \kappa) T)
\end{equation}

to avoid \ac{isi} $\varphi_{gg}$ must fulfill the Nyquist criterion:
\begin{equation}
    \varphi_{gg}(\lambda T) =
    \begin{cases}
        E_g & \lambda = 0 \\
        0 & \lambda \in \mathbb{Z} \setminus \{0\}
    \end{cases}
\end{equation}
if $\varphi_{gg}$ fulfills the Nyquist criterion we call $g$ a $\sqrt{\text{Nyquist}}$-
Pulse. This is also called temporal orthogonality.

furthermore to achieve unit gain we get:
\begin{equation}
    \gamma = \frac{1}{E_g}
\end{equation}

the \ac{acf} of the noise is given by:
\begin{equation}
    \phi_{nn}[\kappa] =
    \begin{cases}
        \frac{N_0}{E_g} & \kappa = 0 \\
        0 & \kappa \in \mathbb{Z} \setminus \{0\}
    \end{cases}
\end{equation}
as a result we can model the channel using discrete time AWGN noise with
\begin{equation}
    \sigma_n^2 = \frac{N_0}{E_g}
\end{equation}

\subsubsection{The Nyquist Criterion in Frequency Domain}
The periodic continuation of the spectrum $\abs{G(f)}^2$ of a time function 
$\varphi_{gg}(t)$ which fulfills the first Nyquist Criterion sums up to a constant:
\begin{equation}
    \sum_{\mu=-\infty}^\infty \abs{G(f - \frac{\mu}{T})}^2 = E_g T
\end{equation}

\subsubsection{Bandwidth}
With $R_T = \frac{R_m}{T}$ (datarate) and $\Gamma = \frac{R_T}{B_\text{RF}}$ 
(spectral effiency) we get ($\alpha$ roll-off factor):
\begin{itemize}
    \item Baseband transmission
        \begin{eqnarray}
            B_\text{RF} &=& \frac{1 + \alpha}{2 T} \\
            \Gamma &=& \frac{2 R_m}{1 + \alpha}
        \end{eqnarray}
    \item Carrier-Modulated Transmission
        \begin{eqnarray}
            B_\text{RF} &=& \frac{1 + \alpha}{T} \\
            \Gamma &=& \frac{R_m}{1 + \alpha} 
        \end{eqnarray}
\end{itemize}

\subsection{Eye Pattern}
The depiction of all sample functions of the cyclo-stationary process yields a periodic
pattern, the so called eye diagram. The horizontal opening is an indicator for the
robustness against timing problems, the vertical opening is an indicator for the robustness
against noise.

\section{Detection of the PAM Signal Points}
We consider the discrete time model:
\begin{equation}
    d[k] = a[k] + n[k]
\end{equation}
as we consider uncoded transmission we can do this symbol by symbol and thus drop the 
time index $k$.

\subsection{\acl{map} Strategy}
Decide in favor of the signal point $a \in \mathcal{A}$ which has the largest probability
given the observation of $d$ (\ac{map}):
\begin{equation}
    \hat{a} = \argmax_{a\in \mathcal{A}} \CPr{a}{d}
\end{equation}

\subsection{\acl{ml} Strategy}
For equiprobable symbols the \ac{map} strategy simplifies to the \ac{ml} strategy:
\begin{equation}
    \hat{a} = \argmax_{a \in \mathcal{A}} f_d(d|a)
\end{equation}
for AWGN channels this is equivalent to minimizing the (squared) euclidean distance
between observation $d$ and hypothesis $\hat{a}$.

\section{Bit and Symbol Error Ratio}
The \ac{ser} is given by:
\begin{equation}
    \Pr{\hat{a} \neq a} = \int_{d \notin \mathcal{G}_a} f_d(d|a) \text{d}d
\end{equation}

The average \ac{ser} is given by (assuming equiprobable symbols):
\begin{equation}
    \text{SER} = \frac{1}{M} \sum_{a \in \mathcal{A}} \Pr{\hat{a} \neq a}
\end{equation}

\subsection{Normalized Minimum Squared Euclidean Distance of a Signal Constellation}
The normalized minimum squared euclidean distance of a signal constellation
$\mathcal{A}$ is given by:
\begin{equation}
    d_\text{min}^2 = \frac{E_g}{2 E_b} \min{}_{a_i, a_j \in \mathcal{A}, a_i \neq a_j}
        \abs{a_i - a_j}^2
\end{equation}

\subsection{\acl{ser} of \acl{ask}}
The average \ac{ser} is given by;
\begin{equation}
    \text{SER} = 
        \frac{2 \cdot (M-1)}{M} Q\left(\sqrt{d_\text{min}^2 \frac{E_b}{N_0}}\right)
\end{equation}

\subsection{\acl{ser} for general \acl{pam} schemes}
An upper bound on the \ac{ser} of digital \ac{pam} is given by:
\begin{equation}
    \text{SER} \leq N_\text{min} Q\left(\sqrt{d_\text{min}^2 \frac{E_b}{N_0}}\right)
\end{equation}
with $N_\text{min}$: the average number of nearest-neighbor signal points.

\subsection{Approximation of the Bit-Error-Rate}
Assuming gray labeling and a reasonably low error rate we arrive at:
\begin{equation}
    \text{BER} \approx \frac{N_\text{min}}{\log_2(M)} Q\left(\sqrt{d_\text{min}^2 \frac{E_b}{N_0}}\right)
\end{equation}

