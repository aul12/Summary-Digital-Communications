\chapter{Introduction}
\section{Definitions}
\subsubsection{Communication}
Communication is the activity of conveying information.

\subsubsection{Channel}
The channel is the part of the communication system
that one is either unwilling or unable to change.

\subsubsection{Transmitter/Receiver Resources}
\begin{itemize}
    \item Signal Power
    \item Bandwidth
    \item Processing and Storage Capabilities
\end{itemize}

\section{Digital Communications}
\subsection{Advantages}
\begin{itemize}
    \item Higher power efficiency
    \item Flexible trade off between power- and bandwidth-effienciency
    \item Signal regeneration is possible
\end{itemize}

\section{Channel-Models}
\subsection{General RF-Model}
\begin{enumerate}
    \item $s_\text{RF}(t)$ is transformed to $r_\text{RF}'(t)$ by attenuation and
        disturbances in the channel
    \item $r_\text{RF}'(t)$ is transformed to $r_\text{RF}(t)$ by the amplifier
        and passband filter of the receiver
\end{enumerate}

\subsection{Equivalent Linear Model in ECB-Domain}
\begin{enumerate}
    \item The signal $s(t)$ is transformed by $h_C(t) \laplace H_C(f)$ to $\tilde{r}(t)$
    \item Additive noise is added to form $r(t) = \tilde{r}(t) + n(t)$
\end{enumerate}

\subsection{Noise}
\subsubsection{White Gaussian Noise}
\begin{itemize}
    \item The noise process $n(t)$ is stationary, i.e. its statistical properties
        do not change over time
    \item The \ac{pdf} at any time instance $t$ is a Gaussian
    \item The noise has constant \ac{psd}:
        \begin{equation}
            \Phi_{nn}(f) = N_0\ \forall f
        \end{equation}
        in time domain: the autocorrelation is:
        \begin{equation}
            \phi_{nn}(t) = N_0 \delta(t)
        \end{equation}
\end{itemize} 

\subsubsection{Colored Gaussian Noise}
Same properties as white gaussian noise but the \ac{psd} is not constant/the autocorrelation
shows temporal dependencies.

\subsubsection{Whitening Filter}
The linear filter
\begin{equation}
    H_W(f) =
    \begin{cases}
        \sqrt{\frac{N_0}{\Phi_{nn}(f)}} e^{j \phi(f)} & \abs{f} \leq \frac{B_\text{RF}}{2} \\
        \text{arbitrary} & \text{else}
    \end{cases}
\end{equation}
transform noise with psd $\Phi_{nn}(f)$ and (baseband) bandwidth $B_\text{RF}$ to
white noise in the baseband.

By implementing this filter as part of our receiver we can guarantee white noise with
the downside of linear distortion of the receive signal. 
Thus we can restrict our derivations to white noise and linear distortions instead of considering both issues.

\subsection{Transmission Media}
\subsubsection{Coaxial Cables}
The transfer function can be approximated by:
\begin{equation}
    H_C(f) \approx 10^{a_\text{dB} l \sqrt{\frac{f}{f_N}} \frac{1}{20}} e^{-j 2\pi \frac{f l}{c_r}}
\end{equation}
with:
\begin{itemize}
    \item $l$: Cable length
    \item $a_\text{dB}$: attenuation per kilometer (in dB) at $f_N$
    \item $f_N$: normalization frequency
    \item $c_r$: propagation speed. Given the electrical properties of the conductor
        it can be calculated by:
        \begin{equation}
            c_r = \frac{c_0}{\sqrt{\varepsilon_r \mu_r}}
        \end{equation}
\end{itemize}

This transfer function is in general non-flat over the transmission signal bandwidth.

\subsubsection{Symmetric Wire Pairs}
The transfer function can be approximated by
\begin{equation}
    \abs{H_C(f)} = 10^{-a_\text{dB}(f) \cdot l / 20}
\end{equation}
with frequency depended attenuation defined by:
\begin{equation}
    a_\text{dB} = a_0 + \sum_{i=1}^n a_i {\left(\frac{f}{f_0}\right)}^{b_i}
\end{equation} 

\subsubsection{Radio Transmission}
\paragraph{Pure Propagation Loss}
Only the attenuation over the channel is considered

\paragraph{Multipath Propagation}
At the receiver multiple scaled and delayed copies of the signal are received:
\begin{equation}
    h_c(t) = \sum_{i=1}^L c_i \delta(t - T_i)
\end{equation}
with:
\begin{itemize}
    \item $c_i$: complex gain factor of the $i$-th propagation path
    \item $T_i$: delay of the $i$-th propagation path
\end{itemize}

thus a frequency depedent attenuation occurs.

\section{Basics on Digital Communications}
\subsection{Transmitter}
The transmitter performs a mapping of an abstract, binary sequence of source symbols
$<q[l]>, l \in \mathbb{Z}, q[l] \in \{0, 1\}$ onto a physical transmit signal $s(t)$.

We assume binary, equal-probable symbols from a memoryless source (can be achieved by using
channel coding).

\subsection{Receiver}
The receiver performs the inverse mapping of the transmitter.

\subsection{Optimality Criteria in Digital Communications}
\subsubsection{Power Efficiency}
The power efficiency is given by the minimal \ac{snr} required to achieve a given \ac{ber}.

High power effiency $\Rightarrow$ small required \ac{snr}. 

\subsubsection{Bandwidth Effiency}
The bandwidth effiency is given by the datarate per bandwidth:
\begin{equation}
    \Gamma = \frac{R_T}{B_\text{RF}}
\end{equation}

High bandwidth effiency $\Rightarrow$ large $\Gamma$.

\subsubsection{Other Criteria}
\begin{itemize}
    \item Complexity
    \item Delay
\end{itemize}

\subsubsection{Trade-off}
It is not possible to optimize both criteria at the same time, a trade off must be found.
The limit for optimization is given by the channel capacity:
\begin{equation}
    C_T = B_\text{RF} \log_2 \left(1 + \frac{S}{R}\right)
\end{equation}
in the best case the channel capacity is equal to the data rate, in this case the limit
is given by:
\begin{equation}
    \frac{E_b}{N_0} = \frac{2^\Gamma -1}{\Gamma}
\end{equation}
