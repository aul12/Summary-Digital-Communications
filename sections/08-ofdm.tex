\chapter{\acl{ofdm}}
\section{Basic Principles of \acl{ofdm}}
We are still considering a linear, dispersive channel with
\begin{equation}
    h_C(t) \laplace H_C(f)
\end{equation}
and \ac{awgn}.

\subsubsection{Frequency Multiplexing}
    The channel has bandwidth $B_\text{RF}$, thus the minimal symbol duration is $T=\frac{1}{B_\text{RF}}$. 

The main idea is to divide the transmission band into $D$ subbands of width:
\begin{equation}
    B_S = \frac{B_\text{RF}}{D}
\end{equation}
the smallest symbol duration for each subband is then given by:
\begin{equation}
    T_S = \frac{1}{B_S} = D T
\end{equation}

in \ac{ecb} domain the transmission band is defined to be at:
\begin{equation}
    f \in \left[-\frac{B_S}{2}, B_\text{RF} - \frac{B_S}{2}\right]
\end{equation}
thus the reference frequency $f_0$ and the center frequency differ. But the 
carrier frequency in the $d$-th subband is defined easily as:
\begin{equation}
    f_{c,d} = d B_s = \frac{d}{T_s}
\end{equation}
for $d \in \{0, \ldots, D-1\}$.

This scheme is called \ac{ofdm}, \ac{dmt} or \ac{mcm}.

\subsubsection{\acl{ofdm} Transmitter}
The naive implementation consisting of $D$ parallel \ac{pam} transmitters/receivers
is to complex, thus some simplifications are made:
\begin{itemize}
    \item Instead of using multiple modulators each with it's own frequency multiple
        pulse shapes are used (identical to how it is done in the general digital
        communications scheme). These pulses are orthogonal.
    \item The pulse shapes $g_j(t)$ are expected to be composed of $D$ chips,
        thus every chip has a duration of $T$.

        Using the chip weights $w_{l,j}$ we can get the pulse shapes as
        \begin{equation}
            g_j(t) = \sum_{l=0}^{D-1} w_{l,j} g_T(t - lT)
        \end{equation}
        i.e. the pulses $g_j(t)$ are \ac{pam} signals.

        The complete transmit signal is given by:
        \begin{eqnarray}
            s(t) &=& \sum_{j=0}^{D-1} \sum_{k_s} a_j[k_s] \sum_{l=0}^{D-1}
                w_{l,j} g_T(t-lT - k_s T_s) \\
                &=& \sum_k x[k] g_T(t-kT)
        \end{eqnarray}
        with 
        \begin{equation}
            x[k] = x_{k \mod D}[\floor{\frac{k}{D}}]
        \end{equation}
        and 
        \begin{equation}
            x_l[k_s] = \sum_{j=0}^{D-1} w_{l,j} a_j[k_s]
        \end{equation}

        the \ac{ofdm} signal is thus a \ac{pam} signal, computed over blocks.
\end{itemize}

\subsubsection{Block Transmission}
In summary we get a block transmission, \ac{isi} is still present, but only within
the block.

We use an whitened matched filter to achieve white noise, but \ac{isi} is still
possible. There are multiple strategies to compensate for \ac{isi} between blocks:
\begin{itemize}
    \item
        After each block $L$ zeros are transmitted, this clears the channel memory
        (as it is an LTI system)
    \item
        Instead of an all-zero word any other fixed word can be used. This is known
        as unique-word \ac{ofdm}, but in general more complex than zero padding.
    \item
        The most common strategy is partial cyclic repetition of the transmit block,
        thereby parts of the signals are appended. This yields a cyclic convolution
        with the channel impulse response
\end{itemize}

The block processing consists of:
\begin{enumerate}
    \item Given the $D$ data symbols $a_j[k_s], j \in \{0, \ldots, D-1\}$ of a block
        at time index $k_s$ the $D$ channel symbols $x_l[k_s], l \in \{0, \ldots, D-1\}$
        are calculated
    \item A cyclic prefix, the guard interval, of $D_0 \geq L$ symbols is introduced
    \item The entire block of $D+D_0$ symbols is serially transmitted over the channel
    \item At the receiver the guard interval is discarded a block of $D$ symbols
        $y_l[k_s], l \in \{0, \ldots, D-1\}$ are calculated
    \item From the block of channel output symbols $D$ detection symbols $d_j[k_s],
        j \in \{0, \ldots, D-1\}$ are introduced
\end{enumerate}

In matrix/vector notation:
\begin{itemize}
    \item The block of the $D$ transmit symbols is given by:
        \begin{equation}
            x[k_s] = \left[x_0[k_s], \ldots, x_{D-1}[k_s]\right]
        \end{equation}
    \item The block of receive symbols is given by:
        \begin{eqnarray}
            y[k_s] &=& \left[y_0[k_s], \ldots, y_{D-1}[k_s]\right] \\
                &=& x[k_s] \cdot H_C + n[k]
        \end{eqnarray}
        with $n[k]$ a vector of \ac{awgn}. The matrix $H_C$ is given by:
        \begin{equation}
            H_C = \begin{pmatrix}
                h[0] & h[1] & \cdots & h[L] & & 0 \\
                0 & h[0] & \ddots & & \ddots\\
                  & \ddots & \ddots & \ddots & & h[L] \\
                h[L] & \cdots & 0 & \ddots & \ddots & \vdots \\
                \vdots & \ddots & & & \ddots & h[1] \\
                h[1] & \cdots & h[L]  & & & h[0]
            \end{pmatrix}
        \end{equation}
\end{itemize}
as $H_C$ is not diagonal \ac{isi} occurs, we need to diagonalize $H_C$. For
this we find to eigenvectors, for circulant matrices the matrix of eigenvectors
(modal matrix) is given by:
\begin{equation}
    W_D = \frac{1}{\sqrt{D}} \begin{pmatrix}
        1 & 1 & 1 & \cdots & 1 \\
        1 & e^{-j\frac{2 \pi}{D}} & e^{-j \frac{2 \pi}{D} 2} & \cdots & e^{-j \frac{2 \pi}{D} (D-1)} \\
        1 & e^{-j \frac{2 \pi}{D} 2} & \text{\reflectbox{$\ddots$}} & & \vdots \\
        \vdots & \vdots \\
        1 & e^{-j \frac{2 \pi}{D} (D-1)} & \cdots & \cdots & e^{-j \frac{2 \pi}{D} {(D-1)}^2}
    \end{pmatrix}
\end{equation}
this shows that we can filter the signal on the transmitter side with the inverse modal
matrix, transmit the signal and filter it on the receiver side with the modal matrix
to yield an \ac{isi} free end to end signal.

We can also describe the protocol in frequency domain:
\begin{enumerate}
    \item The data symbols $a_d[k_s]$ are interpreted to be frequency-domain symbols
    \item The inverse DFT of $a[k_s] = \left[a_0[k_s], \ldots, a_{D-1}[k_s]\right]$
        is calculated to obtain the time-domain symbols:
        \begin{equation}
            x[k_s] = a[k_s] W_D^{-1}
        \end{equation}
    \item The guard interval is inserted
    \item The block is serially transmitted over the channel
    \item At the receiver the guard interval is ignored and the receive block is extracted
    \item The DFT of $y[k_s]$ is calculated to obtain
        \begin{equation}
            d[k_s] = y[k_s] W_D
        \end{equation}
    \item The detection symbols $d_j[k_s]$ can be again interpreted to be in frequency domain
\end{enumerate}

\subsubsection{End-to-End Model of \acl{ofdm}}
Like derived above the \ac{isi} channel is transformed into $D$ parallel, independent
channels with individual scaling. The scaling factors are the eigenvalues. The noise
variance is given by:
\begin{equation}
    \sigma_{n,k}^2 \approx \Phi_{nn}(e^{j 2 \pi \frac{k}{D}})
\end{equation}


\subsubsection{Average \acl{psd} of \acl{ofdm} signals}
It can be shown that the average \acl{psd} is given by:
\begin{equation}
    \bar{\Phi}_{ss}(f) = \left(\frac{T_s}{D} \sum_{\mu = -\infty}^\infty
        \sigma^2_{a, \mu \mod D} \text{si}^2 (\pi (f T_s - \mu \frac{D + D_0}{D}))\right)
        \abs{H_T(f)}^2
\end{equation}

\section{Optimization of \acl{ofdm}}
The individual channels are all assumed to be flat. Still there can be a difference in
attenuation and thus \ac{snr} over the different channels. To improve performance
coded transmission can be used.

Regarding coding two scenarios need to be differentiated:
\begin{itemize}
    \item Broadcast scenario: the channel is different per user thus a general scheme,
        \ac{cofdm} must be used
    \item Point-to-Point scenario: the channel transfer function is known and thus the
        transmission rate over the individual channels can be optimized using so-called
        loading algorithms 
\end{itemize}

For a loading algorithm we demand that the total transmit power and the total data
rate should be fixed.

\subsubsection{Optimization}
We will minimize the uncoded error probability under the given constraints. The
\ac{ser} for a single channel is given by:
\begin{equation}
    \text{SER}_j = N_{\text{min},j} Q\left(\sqrt{\text{SNR}_{0,j}}\right)
\end{equation}
with
\begin{equation}
    \text{SNR}_{0,j} = \frac{\abs{\lambda_j}^2 V_j^2}
        {\sigma_{n,j}^2 / 2}
\end{equation}
with $V_j$ being a scaling factor used for adjusting the power.


For 4-\ac{qam} the factor $N_{\text{min},j} \approx 4$ is constant. Additionally we can
see that lowest \ac{ser} is achieved for a constant error rate over the channels.
For this we define the \ac{snr} as a function of the indivdual \ac{snr}s as:
\begin{equation}
    \text{SNR} = \frac{\sigma_a^2}{2} \text{SNR}_0
\end{equation}
furthermore we demand (derived from the constant error rate):
\begin{equation}
    \text{SNR}_{0,j} = \frac{V_J^2}{Q_j/2} \stackrel{!}{=} \text{SNR}_0
\end{equation}
with the definition $Q_j = \sigma_{n,j}^2 / \abs{\lambda_j}^2$. Thus we want to optimize
$\text{SNR}_0$ under the given constraints.

When relaxing the constraint of integer constellations we can use Langrangian optimization
which will lead to:
\begin{eqnarray}
    R_j &=& \frac{R_T}{D} + \frac{1}{D} \log_2 \left(\frac{\prod_{l=0}^{D-1}
        \frac{\sigma_{n,l}^2}{\abs{\lambda_l}^2}}{
        {\left(\frac{\sigma_{n,j}^2}{\abs{\lambda_j}^2}\right)}^D}\right) \\
    S_j &=& \frac{S_T}{D}
\end{eqnarray}

\section{Comparison of Single-and Multicarrierer Modulation}
\begin{itemize}
    \item Single Carrier Modulation is time invariant but nonlinear
    \item Multi Carrier Modulation is time variant but linear
\end{itemize}

The \ac{snr} of optimally loaded \ac{snr} is given by:
\begin{equation}
    \text{SNR}^\text{(OFDM)} = \exp\left(T \int_{-\frac{1}{2T}}^\frac{1}{2T} 
        \log\left(\tilde{\text{SNR}}(e^{j 2 \pi f T})\right)\text{d}f\right)
\end{equation}
thus \ac{ofdm} and \ac{dfe} have potentially the same performance.
