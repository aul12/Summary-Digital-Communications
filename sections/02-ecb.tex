\chapter{Equivalent Complex Baseband Signals}
\section{Definitions}
\subsection{Bandpass Signal}
Spectrum only exists for
\begin{equation}
    f_0 - \frac{B}{2} < \abs{f} < f_0 + \frac{B}{2}
\end{equation}
with
\begin{itemize}
    \item $f_0$: center frequency (in general not the same as the carrier frequency)
    \item $B$: one sided bandwidth
\end{itemize}

\subsection{Baseband Transmission}
A bandpass transmission with $f_0=0$.

\subsection{Carrier}
A carrier waveform is given by:
\begin{equation}
    c(t) = \sqrt{2} A_\text{RMS} \cos(2 \pi f t + \varphi)
\end{equation}
this harmonic oszillation is characterized by three properties:
\begin{itemize}
    \item $A_\text{RMS}$: RMS-Amplitude
    \item $f$: Frequency
    \item $\varphi$: Phase
\end{itemize}

The momentarily frequency is directly related to the time dependent phase:
\begin{equation}
    f_M(t) = \frac{1}{2\pi} \frac{\text{d}}{\text{d}t} \varphi(t)
\end{equation}

thus it is only necessary to consider one of the two signals for an arbitrary bandpass
signal:
\begin{equation} \label{eq:02:srf}
    s_\text{RF}(t) = \sqrt{2} a_M(t) \cos(2\pi f_0 t + \varphi_M(t))
\end{equation}
with
\begin{itemize}
    \item $f_0$: reference frequency
    \item $a_M(t)$: amplitude signal
    \item $\varphi_M(t)$: phase signal
\end{itemize}
All signals are real-values. Both $a_M$ and $\phi_M$ are lowpass/baseband signals.

Furthermore we can assume $a_M(t) \geq 0\ \forall t$ and 
$\varphi_M(t) \in (-\pi, \pi]\ \forall t$.

\subsection{Analytical Signal}
The corresponding analytical signal for a given real valued signal is given by:
\begin{equation}
    x^+(t) \laplace X^+(f) = (1 + \text{sgn}(f)) X(f)
\end{equation}
thus all negative frequencies are surpressed and positive frequencies are amplified.

\subsection{Equivalent Complex Baseband Signal}
Using equation \ref{eq:02:srf} and eulers-formula we come to:
\begin{equation}
    s_\text{RF}(t) = \sqrt{2} a_M(t) \cos(2\pi f_0 t + \varphi_M(t))
        = \sqrt{2} \Re\{s(t) e^{j 2 \pi f_0 t}\}
\end{equation}
with
\begin{equation}
    s(t) = a_M(t) e^{j \varphi_M(t)}
\end{equation}
thus we can represent every baseband signal by an \ac{ecb} signal.

This can also be seen as a the (complex) envelope of the RF signal.

\subsection{ECB to RF transformation}
The transformation is given by shifting/modulating and surpressing the imaginary parts:
\begin{equation} 
    x_\text{RF}(t) = \frac{1}{\sqrt{2}} \left(x(t)e^{j 2 \pi f_0 t} + x^*(t) e^{-j 2 \pi f_0 t}\right)
\end{equation}
or
\begin{equation}\label{eq:02:xrf}
    X_\text{RF}(t) = \frac{1}{\sqrt{2}} \left(X(f-f_0) + X^*(-(f+f_0)\right)
\end{equation}

\subsection{RF to ECB transformation}
By solving equation \ref{eq:02:xrf} for $X(f)$ using the Hilbert-Transform we arrive at:
\begin{equation}
    X(f) = \frac{1}{\sqrt{2}} (1 + \text{sgn}(f + f_0)) X_\text{RF}(f+f_0)
        = \frac{1}{\sqrt{2}} X^+_\text{RF}(f+f_0)
\end{equation}
or
\begin{equation}
    x(t) = \frac{1}{\sqrt{2}} (x_\text{RF}(t) + j\mathcal{H}\{x_\text{RF}(t)\}) 
        e^{-j 2 \pi f_0 t}
    = \frac{1}{\sqrt{2}} x_\text{RF}^+ (t) e^{-j 2 \pi f_0 t}
\end{equation}

\section{The Quadrature Components of ECB Signals}
\subsection{ECB-Representations}
\subsubsection{Polar-Form}
Given an ECB Signal $x$ we can split it into a magnitude and phase term:
\begin{equation}
    x(t) = \abs{x(t)} e^{j \arg\{x(t)\}}
\end{equation}

\subsubsection{Quadrature Components}
On the other hand it is possible to split the signal in real part (inphase component)
and imaginary part (quadrature component):
\begin{equation}
    x(t) = \Re{x(t)} \Im{x(t)} = x_I(t) + j x_Q(t)
\end{equation}

\subsection{Quadrature Upconversion}
Taking the quadrature components for the ECB to RF transformation we get:
\begin{equation}
    x_\text{RF}(t) = \sqrt{2}\left(x_I(t) \cos(2\pi f_0 t) - x_Q(t) \sin(2 \pi f_0 t)\right)
\end{equation}

\subsection{Quadrature Downconversion}
Using the RF to ECB transformation and the definition of the quadrature components we
get:
\begin{eqnarray}
    x_I(t) &=& \frac{1}{\sqrt{2}} \left(x_\text{RF}(t) \cos(2 \pi f_0 t) + 
        \mathcal{H}\{x_\text{RF}(t)\} \sin(2 \pi f_0 t)\right)\\ 
    x_Q(t) &=& \frac{1}{\sqrt{2}} \left(\mathcal{H}\{x_\text{RF}(t)\} \cos(2 \pi f_0 t)
        - x_\text{RF}(t) \sin(2 \pi f_0 t)\right)
\end{eqnarray}

For bandpass signals, that are signals with $X_\text{RF}(f)=0, \abs{f} > 2 f_0$,
the Hilbert-Transform can be replaced by an ideal-low pass filter with cut-off frequency $f_0$.

This leads to
\begin{equation}
    X(f) = \sqrt{2} \text{rect}\left(\frac{f}{2 f_0}\right) X_\text{RF}(f+f_0)
\end{equation}
or
\begin{equation}
    x(t) = \sqrt{2} h_\text{LP}(t) * (x_\text{RF}(t) e^{-j 2 \pi f_0 t})
\end{equation}
with $h_\text{LP} \laplace \text{rect}\left(\frac{f}{2 f_0}\right)$.

Thus a simplified quadrature downconversion is given by:
\begin{eqnarray}
    x_I(t) &=& \sqrt{2} \left(x_\text{RF}(t) \cos(2\pi f_0 t)\right) * h_\text{LP}(t) \\
    x_Q(t) &=& -\sqrt{2} \left(x_\text{RF}(t) \sin(2\pi f_0 t)\right) * h_\text{LP}(t)
\end{eqnarray}

\subsection{LTI Systems and ECB Signals}
Let the real-valued system $h_\text{RF}(t) \laplace H_\text{RF}(f)$ be given. The 
corresponding \ac{ecb}-system $h(t)$ with transfer function $H(f)$ is given by:
\begin{equation}
    h(t) = \frac{1}{2} \left(h_\text{RF}(t) + j \mathcal{H}\{h_\text{RF}(t)\}\right) e^{-j 2 \pi f_0 t}
\end{equation}
or
\begin{equation}
    H(f) = \frac{1}{2} \left(1 + \text{sgn}(f + f_0)\right) H_\text{RF}(f+f_0)
\end{equation}

the real valued system can be recovered from an \ac{ecb} system by:
\begin{equation}
    h_\text{RF}(t) = 2 \Re{h(t) e^{j 2 \pi f_0 t}}
\end{equation}
or
\begin{equation}
    H_\text{RF}(f) = H(f-f_0) + H^*(-(f+f_0))
\end{equation}

\subsubsection{Quadrature Components}
Similar to signals the quadrature components can also be defined for systems:
\begin{eqnarray}
    h(t) = h_I(t) + j h_Q(t)
\end{eqnarray}

The sytem response is then given by:
\begin{equation}
    y(t) = x(t) * h(t) =
        (x_I(t) * h_I(t) - x_Q(t) * h_Q(t)) + j(x_I(t) * h_Q(t) + x_Q(t) * h_I(t))
\end{equation}
thus it incorporated four real-valued LTI systems.

\section{Stochastic Processes}
\subsection{Parameters of Stochastic Processes}
\subsubsection{\acl{pdf}}
At every timestep $t$ the process $\mathbf{x}(t)$ reduces to a random variable
$X=\mathbf{x}(t)$ which can be characterized by its \ac{pdf}.

This \ac{pdf} can be described by properties such as mean, variance and power (sum
of mean and variance).

\subsubsection{\acl{acf}}
The \ac{acf} describes the dependency of the properties over time:
\begin{equation}
    \phi_{xx} (t_1, t_2) = E\{\mathbf{x}(t_1) \cdot \mathbf{x}(t_2)\}
\end{equation}

\subsubsection{\acl{ccf}}
Similarly the cross correlation can be defined:
\begin{equation}
    \phi_{xy} (t_1, t_2) = E\{\mathbf{x}(t_1) \cdot \mathbf{y}(t_2)\}
\end{equation}

\subsection{Wide-Sense Stationary}
A stochastic process is (wide-sense) stationary if:
\begin{itemize}
    \item The mean is constant over time
    \item The \ac{ccf} and thus the \ac{acf} as well depend only on the time difference
\end{itemize}

As the \ac{acf} only depends on the time difference it can be interpreted as a signal,
an important property is the fourier transform which is called the \ac{psd}.

The fourier transform of the \ac{ccf} is called cross power spectral density.

\subsection{LTI Systems and Stochastic Processes}
The following relations hold ($\varphi_{hh}(\tau) = h(\tau) * h*(-\tau)$):
\begin{eqnarray}
    \phi_{xy}(\tau) &=& \phi_{xx}(\tau) * h(-\tau) \\
    \Phi_{xy}(f) &=& \Phi_{xx}(f) \cdot H^*(f) \\
    \phi_{yx}(\tau) &=& \phi_{xx}(\tau) * h(\tau) \\
    \Phi_{yx}(f) &=& \Phi_{xx}(f) \cdot H(f) \\
    \phi_{yy}(\tau) &=& \phi_{xx}(\tau) * \varphi_{hh}(\tau) \\
    \Phi_{yy}(f) &=& \Phi_{xx}(f) \cdot \abs{H(f)}^2
\end{eqnarray}

\subsection{Sampling of Stochastic Processes}
The time discrete \ac{acf} is given by sampling the continuous \ac{acf}.

\subsection{Cyclo-Stationary Stochastic Processes}
For cyclo-stationary stochastic processes the characteristic properties are not constant
over time but fluctuate periodically.

For such processes the average \ac{acf} is of interest:
\begin{equation}
    \bar{\phi}_{xx}(\tau) = \frac{1}{T} \int_0^T \varphi_{xx}(t+\tau, t) \text{d}t
\end{equation}
Similarly we can define the average \ac{psd} as the fourier transform of the average
\ac{acf}.

\subsection{Stochastic Processes in the \acl{ecb}}
\subsubsection{Mean}
The mean in the \ac{ecb} domain is given by
\begin{equation}
    m_x(t) = \frac{1}{\sqrt{2}} m_{x_\text{RF}} e^{-j 2 \pi f_0 t}
\end{equation}

If the RF-Process has non zero mean the \ac{ecb} process has non constant mean and is thus
not stationary.

\subsubsection{\acl{acf}}
The \ac{acf} in the \ac{ecb} domain is given by
\begin{equation}
    \phi_{xx}(\tau) = \left(\phi_{x_\text{RF} x_\text{RF}} (\tau) + j 
        \mathcal{H}\{\phi_{x_\text{RF} x_\text{RF}} (\tau)\}\right)
        e^{-j 2 \pi f_0 \tau}
\end{equation}

or
\begin{equation}
    \Phi_{xx}(f) = (1 + \text{sgn}(f+f_0)) \Phi_{x_\text{RF} x_\text{RF}}(f+f_0)
\end{equation}

\subsubsection{Existenz of RF-Signal}
A complex valued, wide-sense stationary process $\mathbf{x}(t)$ is only equivalent
to a real valued wide-sense stationary process $\mathbf{x}_\text{RF}(t)$ if:
\begin{itemize}
    \item The mean is zero
    \item Real and imaginary part have the same \ac{acf}
    \item The \ac{ccf} between real and imaginary part is odd
\end{itemize}

