\chapter{Variants of \acl{pam} Transmission Schemes}
\section{Offset \acl{qam}}
\subsection{Crest Factor of \acl{pam} Transmit Signals}
The crest factor of an \ac{ecb} signal $s(t)$ with average power $S_S$ is given by:
\begin{equation}
    \zeta = \frac{\max_{\forall t} \abs{s(t)}}{\sqrt{S_S}}
\end{equation}

The \ac{par} can be defined as:
\begin{equation}
    \text{PAR} = \zeta^2
\end{equation}

Due to the modulation the RF signals has a $\sqrt{2}$/$2$ lower crest factor/\ac{par}
than the \ac{ecb} signal. We will only consider the \ac{ecb} signal.

We can also define the crest factor of the constellation:
\begin{equation}
    \zeta_a = \frac{\max_{a \in \mathcal{A}} \abs{a}}{\sqrt{E\{\abs{a}^2\}}}
\end{equation}
and of the basic pulse shape:
\begin{equation}
    \zeta_g = \frac{\max_{\forall t} \sum_k \abs{g(t-kT)}}{\sqrt{E_g /T}}
\end{equation}
we can show that:
\begin{equation}
    \zeta \leq \zeta_a \cdot \zeta_g
\end{equation}

\subsection{Characterization of Power Amplifiers}
The distortion of an amplifier can be modeled by:
\begin{equation}
    s'(t) = A\left(\frac{\abs{s(t)}}{s_0}\right) 
        e^{j \left(\arg(s(t)) + P\left(\frac{\abs{s(t)}}{s_0}\right)\right)}
\end{equation}
with:
\begin{itemize}
    \item $s_0$: Reference amplitude
    \item $A$: AM/AM characteristics
    \item $P$: AM/PM characteristics
\end{itemize}

\subsection{Offset \acl{qam}}
The \ac{qam} transmitter is modified to delay the signal in the quadrature branch
by $T/2$. The purpose is to lower the fluctuations of the envelope and thus achieve
a lower crest factor:
\begin{equation}
    \zeta_g = \frac{\max_{\forall t} \sqrt{{\left(\sum_k \abs{g(t-kT)}\right)}^2
        + {\left(\sum_k \abs{g(t-kT - T/2)}\right)}^2}}{\sqrt{2 E_g / T}}
\end{equation}

This results from the fact that the inphase and quadrature part now do not change
at the same time but shifted relative to each other. Thus only the real or the 
imaginary part changes, the phasor trajectory can not cross the origin.

\section{\acl{msk}}
Using offset \ac{qam} zero crossings are avoided but the envelope is not constant.
For 4-\ac{qam} with the pulse shape:
\begin{equation}
    g(t) = \sqrt{\frac{2 E_g}{T}} \cos\left(\pi \frac{t}{T}\right) 
        \text{rect}\left(\frac{t}{T}\right)
\end{equation}
we get a constant envelope, this is called \ac{msk}.

Instead of viewing this as a special case of offset \ac{qam} we can also see this as
a \ac{pam} scheme applying two different pulses for the inphase component ($g(t)$) and
the quadrature component ($g(t-T/2)$).

The average \ac{psd} is given by:
\begin{equation}
    \bar{\Phi}_{ss}(f) = \frac{16 E_g}{\pi^2} \frac{\cos^2(\pi f T)}{{(1-{(2fT)}^2)}^2}
\end{equation}
thus it is not bandlimited.

Instead of specifying the bandwidth as infinity a $B_{X\%}$ bandwidth is given,
i.e. the bandwidth that contains $X\%$ of the transmit power:
\begin{equation}
    \int_{-B_{X\%}}^{B_{X\%}} \bar{\Phi}_{ss}(f) \text{d}f =
        \frac{X}{100} \int_{-\infty}^\infty \bar{\Phi}_{ss}(f) \text{d}f
\end{equation}

For \ac{msk} we get ($R_T = 1/T_b$):
\begin{equation}
    B_{99\%} = 1.11 R_T
\end{equation}
and thus
\begin{equation}
    \Gamma_{99\%} = \frac{R_T}{B_{99\%}} = 0.9 \frac{\text{bit}/\text{s}}{\text{Hz}}
\end{equation}

For normal \ac{qam} with $\alpha=0$ we have $\Gamma = 2$, thus the spectral efficiency of
\ac{msk} is approximatly half the spectral efficiency of \ac{qam}. 

\subsection{Equivalenz Between \acl{msk} and Binary Frequency Modulation}
\ac{msk} can also be seen as binary frequency modulation with double the frequency
(to achieve the same data rate).

\section{\acl{gmsk}}
\ac{msk} guarantees a constant envelop but yields an infinite bandwidth and thus poor
spectral effiency. This is a result of the discontinous instantaneous frequency.
To solve this issue the instantaneous frequency is smoothened in
\ac{gmsk} using a gaussian filter, thus no frequency hard keying occurs.

The filter is given by:
\begin{equation}
    h_G(t) = \frac{1}{\sqrt{2 \pi T_b^2 \tau^2}} e^{-\frac{(t/T_b)^2}{(2\tau)^2}}
\end{equation}
with $\tau = \frac{\sqrt{\log(2)}}{2 \pi B_b T_b}$ or
\begin{equation}
    H_G(f) = e^{-(f/B_b)^2 \log(2) /2}
\end{equation}
$B_B$ is the bandwidth parameter that can be tuned, GSM chooses $T_b B_b = 0.3$.

As the operations are linear $h_G$ can be combined with $g_F$.

\subsection{Power and Bandwidth Efficiency of \acl{gmsk}}
As \ac{gmsk} is a nonlinear scheme we can not directly apply the calculations of the
previous chapter, instead we have to do numerical simulations.

\subsection{Receiver}
Optimal receivers for \ac{gmsk} are not part of the course. Suboptimal receivers can be
designed when considering \ac{gmsk} as \ac{msk} which is a special case of offset
\ac{qam}, thus a \ac{pam} receiver can be used.

\section{Non-Coherent PAM Schemes}
For frequency/phase errors the receive signal is given by:
\begin{equation}
    r(t) = s(t) \cdot e^{j (2 \pi f_\Delta t + \varphi_\Delta)}
\end{equation}
with
\begin{itemize}
    \item $f_\Delta$: The frequency error
    \item $\varphi_\Delta$: The phase error
\end{itemize}

\subsection{PAM Scheme with phase error}
The end-to-end discrete time receive signal is given by:
\begin{equation}
    d[k] = a[k] e^{j \varphi_\Delta} + n[k]
\end{equation}

\subsection{Non-Coherent Demodulation of Unipolar ASK}
The phase can not convey any information, thus all information needs to be in the
amplitude. This is the case for unipolar ASK. The simplest form of unipolar \ac{ask}
is binary unipolar \ac{ask} or \ac{ook}.

In non coherent reception the decision is based on $\abs{d}$, thus the decision
boundary is a circle.

\subsubsection{Error Probability}
For the error probability we need to find the radius $\rho$ which minimizes the
bit error rate (which is equal to the symbol error rate). For this the individual errors
need to be calculated individually, as the problem is not symmetric.
From the individual error rates the average error rate can be calculated, from
this the minimum error rate can be calculated as a function of the \ac{snr}.

In practice $\rho=1$ is a good choice. An upper bound of the error probability
using $\rho=1$ is given by:
\begin{equation}
    \text{BER} < e^{-\frac{E_b}{2 N_0}}
\end{equation}

\subsubsection{Non-Coherent Demodulation of $M$-ary Unipolar ASK}
The error rate can be generalized to $M$-ary ASK:
\begin{equation}
    \text{SER} < e^{-\frac{E_g}{N_0}} = e^{-d_\text{min}^2 \frac{E_b}{2 N_0}}
\end{equation}
when assuming Gray labeling the \ac{ber} is given by:
\begin{equation}
    \text{BER} \approx \frac{1}{\log_2(M)} \text{SER}
\end{equation}

\subsubsection{Comparison with Coherent Demodulation}
The power effiency of non coherent \textbf{uni}polar schemes is up to
6dB worse than the effiency of coherent \textbf{bi}polar schemes. When comparing
unipolar coherent with non coherent transmission nearly no difference exists.

\subsection{\acl{dpsk}}
Options to reduce phase ambiguity:
\begin{itemize}
    \item (Fixed) synchronization sequence
    \item Differential Encoding
\end{itemize}

\subsubsection{Differential $M-$-ary PSK}
We can define two different (but equivalent) transmitter structures. 
$b[k] \in \{0, \ldots, M-1\}$ is a single symbol represented by an integer.

\begin{itemize}
    \item Phase based:
        \begin{equation}
            a[k] = a[k-1] \cdot e^{j 2 \pi \frac{b[k]}{M}}
        \end{equation}
    \item Index based:
        \begin{equation}
            a[k] = e^{j 2 \pi \frac{m[k]}{M}}
        \end{equation}
        with
        \begin{equation}
            m[k] = (m[k-1] + b[k]) \mod M
        \end{equation}
\end{itemize}

From the two transmitter architectures two (not equivalent) receiver architectures can
be derived.

\subsubsection{\acl{dcpsk} Demodulator}
The \ac{dcpsk} demodulator consists of normal $M$-ary \ac{psk} receiver followed by
differentiation.

The \ac{ser} is given by:
\begin{equation}
    \text{SER}_\text{DCPSK} = 2 \text{SER}_\text{PSK} - 1.5 {\text{SER}_\text{PSK}}^2
\end{equation}

\subsubsection{\acl{dpsk} Demodulator}
The DPSK demodulator consists of a matched filter followed by the differentiantial
demodulation and then a $M$-PSK threshold device.

The \ac{ber} for binary \ac{dpsk} is given by:
\begin{equation}
    \text{BER}_\text{DPSK} = \frac{1}{2} e^{-\frac{E_b}{N_0}}
\end{equation}
\ac{dpsk} has roughly the same power efficiency as coherent, binary \ac{psk}.

For non binary \ac{dpsk} we get:
\begin{equation}
    \text{SER}_\text{DPSK} < 1 + Q_M(X_1, X_2) - Q_M(X_2, X_1)
\end{equation}
with
\begin{eqnarray}
    X_1 &=& \sqrt{\log_2(M)\left(1-\sin\left(\frac{\pi}{M}\right)\right)\frac{E_b}{N_0}} \\
    X_2 &=& \sqrt{\log_2(M)\left(1+\sin\left(\frac{\pi}{M}\right)\right)\frac{E_b}{N_0}} 
\end{eqnarray}

for $M$ and $E_b/N_0$ sufficiently large this can be simplified to:
\begin{equation}
    \text{SER}_\text{DPSK} \approx 2 Q\left(\sqrt{d_\text{min}^2 \frac{E_b}{N_0} / 2}\right)
\end{equation}
which is a loss of 3dB when compared to coherent \ac{psk}.
